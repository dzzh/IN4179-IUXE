\documentclass[a4paper]{article}

\makeatletter
\renewcommand\paragraph{\@startsection{paragraph}{4}{\z@}%
  {-3.25ex\@plus -1ex \@minus -.2ex}%
  {1.5ex \@plus .2ex}%
  {\normalfont\normalsize\bfseries}}
\renewcommand\subparagraph{\@startsection{subparagraph}{5}{\z@}%
  {-3.25ex\@plus -1ex \@minus -.2ex}%
  {1.5ex \@plus .2ex}%
  {\normalfont\normalsize\bfseries}}
\makeatother

\usepackage{fullpage}
\usepackage{graphicx}
\usepackage{wrapfig}
\usepackage{enumitem}
\usepackage[font=small,labelfont=bf]{caption}
\graphicspath{{./images/}}
\pagenumbering{arabic}
\begin{document}

\begin{titlepage}

\begin{center}

Delft University of Technology\\
Faculty of Electrical Engineering, Mathematics and Computer Science\\[3cm]
\huge \bf{Developing a Child-Driven \\ Online Learning Environment \\ for Third-World Countries}\\[15cm]
\end{center}

\large \noindent 
Joris Albeda, Chiel Huurdeman, Elmer Jacobs and Zmitser Zhaleznichenka\\
\{J.Albeda, C.Huurdeman, E.Jacobs, D.V.Zhaleznichenka\}@student.tudelft.nl\\

\noindent\today

\end{titlepage}

\setcounter{secnumdepth}{3}

\abstract \emph{This is the final report for IN4179 Intelligent User Experience Engineering course that describes an online learning environment for children in third-world countries that is based on the ideas of Sugata Mitra and employs a virtual avatar of a teacher communicating with the children to facilitate the educational process. Thus we try to exploit so-called "granny effect" to let young children learn by themselves in groups using the computers connected to the Internet.}

\section{Introduction}

There are many problems related to the education process in third-world countries. One of the most important ones is lack of experienced teachers and good study materials. This problem attracts the attention of many specialists in education who suggest their solutions aimed to decrease the role of a teacher and force the children to learn themselves using the achievements of the digital era.

This project was inspired by "The child-driven education" TED Talk\footnote{http://www.ted.com/talks/sugata\_mitra\_the\_child\_driven\_education.html} given by a world-recognised Indian researcher Sugata Mitra who has conducted a "Hole in the Wall" project in 1999 and has proven that the little kids from developing countries are able to study themselves with the help of the computers. Since then Sugata Mitra conducted a number of experiments on self-organised computer-assisted learning and obtained very important results. 

%add a sentence describing the results

We try to develop Sugata Mitra ideas by providing a computer assistant that is aimed in stimulating kids to complete the study challenges submitted to the system by the experienced teachers, assign these challenges to the students and perform some evaluations on the given feedback.

The prototype of the application that we built uses the avatar of a teacher that communicates with the students. We have decided to organise the interaction with the students via the avatar to explore the "granny effect" that was also described by Sugata MItra. This effect is the following. It is known that the best way to learn is to teach. When someone tries to explain a studied concept to anyone, he learns it better himself as he has to structure the information in his head and really understand the details of the material before sharing them with others. Thus, it does not really matters who is the listener. If a kid talks to his granny discussing the subjects he learnt at school, he remembers the study material better that if he would not discuss it with anyone. Our avatar motivates the students to discuss and share their progress thus serving as such "granny".

If implemented correctly, the computer-assisted educational systems aimed at self-organised group-based learning can be extremely useful as the potential amount of children who can benefit from the introduction of such a system is very large. Despite all the improvements in the transportation, technology and communication reached during the last decades, many kids still cannot be guided by the experienced teachers during their childhood. As Sugata Mitra points out, "there are places on Earth, in every country, where, for various reasons, good schools cannot be built and good teachers cannot or do not want to go". With our learning environment we strive to apply the ideas of Sugata Mitra to provide a tool that would be beneficial for the schools lacking the experienced teaching.

%outline

\section{OLE Requirements}

While developing the learning environment, we have specified the following set of requirements.

\subsection{General requirements}

\begin{enumerate}
\item The main goal of a project is to develop an online learning environment (OLE) using a virtual assistant that will stimulate self-organised group-based learning for the children of age 9-12 years from all over the world (from later on also referred as students).

\item OLE should consist of challenge database, student interface, teacher interface and a management console.

\item For working with any part of OLE, a computer with internet access is required.
\end{enumerate}

\subsection{Requirements to challenge database}

\begin{enumerate}
\item The challenge database (CD) is used as a storage for the challenges that are submitted by teachers/volunteers and are used in the learning process. Challenges may contain textual data, images, musical files and short video movies.

\item The challenges are intended to be designed in such a way that they motivate students in gaining more knowledge about the subject, or provide a new angle on subjects the student has some experience with.

\item Apart from storing the contents of the challenges, CD should store such information as the number of times the challenge was used and all the feedbacks provided by the students. CD should allow to store translations of the challenges.
\end{enumerate}

\subsection{Requirements to student interface}

\begin{enumerate}
\item Student interface (SI) is a native application for SugarOS environment used in One Laptop Per Child project\footnote{http://laptop.org}. 

\item The goal of SI is to interact with the student group by assigning them with challenges submitted by the teachers/volunteers and gathering feedback on their accomplishment. 

\item SI should be intuitive enough to let users start working with it not having prior computer experience.

\item SI should be able to properly display images, play musical files and show movies, if these media data is included into the contents of a challenge in CD.

\item Interaction is performed via the animated avatar.
\end{enumerate}

\subsubsection{Requirements to avatar}

\begin{enumerate}
\item Avatar design should focus on creating a realistic and trustful person. Most attention should be paid to the facial expressions, as the body language doesn't affect learning process much \cite{Cowell}.

\item The avatar should be adjustable to cope with different types of education-related challenges, i.e. the challenges asking a question with a concise answer need in another presentation and feedback gathering if comparing with process-oriented challenges.

\item The avatar should motivate user to complete the challenge if the user loses his interest in it and motivate to look for additional information, not only answer the questions. It is important to build the motivation part that will not be very annoying.

\item Interaction with the student should be organised either in textual form with pop-up windows or with text-to-speech mechanisms, or with both of them.
\end{enumerate}


\subsection{Requirements to teacher interface}

\begin{enumerate}
\item Teacher interface (TI) is a web application that connects to CD and allows teachers/volunteers to add new challenges, edit and translate existing challenges, analyse the feedback for the complete challenges and use the existing challenges to build the learning plans for SI.

\item TI should include a guide on the creation of challenges. In the guide it should be explained what sorts of challenges are to be added to the system and several examples of the challenges of every type should be demonstrated. Generally, the challenges should be designed to encourage the students to use available online instruments and stimulate group work.

\item TI should allow teachers to add not only the textual information, but also multimedia data for all the media files supported in SI.

\item TI should allow teachers manage the computers of the associated students. This means, that after student laptop connects to the internet, it should automatically download the learning plan submitted by teacher for this student group and stick to this plan when students work with the OLE.

\item TI should have a simulator of SI, so a teacher will be able to see how the avatar behaves in respond to the submitted challenge and do the necessary adjustments.
\end{enumerate}


\subsection{Requirements to the management panel}

\begin{enumerate}
\item Management panel (MP) is designed as a web application allowing to manage data in CD and control SI and TI. 

\item MP should allow to navigate through the challenges available in CD, edit and remove them with ease.

\item MP should allow to manage teacher accounts created to work with TI.
\end{enumerate}

\section{Scenarios}

\subsection{Adding a challenge}

Lisa is a 30 year-old elementary school teacher from the Netherlands. She recently taught her own class about the sun and she thinks this would interest third world children as well. She heard about the OLE through a colleague and wants to contribute to it. She decides to add a challenge about the sun to the challenge database. 

She goes to the TI website and creates a teacher account. After logging in on the website she clicks the "Add Challenge" button. This takes her to the challenge entry form. She enters a title for the challenge and fills in the first question/assignment the avatar will pose to the children. She figures a good way to get the children's interest is to ask the question; "What is the weight of the sun?". She enters the question on the form and clicks "Add question". She hopes this will grab the children's attention so she enters a more difficult follow-up question: "Why do plants need sunlight?", and clicks "Add question" again. She thinks this will keep children busy for quite a while so she decides to leave this challenge with two questions. She clicks the "Done" button and is returned to the main page. 

\subsection{Carrying out a challenge}

Raja is a 11-years-old boy who lives in India. He lives in a village with very little work opportunity, and no educational facilities. Raja is a clever boy who asks questions about the world around him. He hopes one day to travel to a richer part of India and earn money for his family. But without a school he will not be able to get the proper education he needs.

Then one day a computer is brought to his village, and he hears that this device is meant to help children to learn. Today he is standing in front of the the computer and tries out the new system with two of his friends. The system is now issuing the first challenge, which happens to be about the sun.

The screen shows the avatar, a woman asking him to find out the weight of the sun. Raja's friend tells him to try out the browser. Raja opens the browser, and searches the internet for information about the sun. They find not only the weight of the sun, but also how big it is and what it looks like from up close. They give the answer to the avatar, who reacts delightedly, and asks them to find out why plants need sunlight. They find out together how photosynthesis works, explaining to each other the parts they don't understand.Finally they give an answer to the avatar, who praises them for their efforts, tells them that they finished this challenge and suggests to continue with a new one. 

\subsection{Carrying out a challenge with sibling-based learning} 

Saanvi is a 9-years-old Indian girl. She lives on the Indian countryside with her parents and an elder brother, who is 12-years-old. She has had very little education and does not speak English. One day she decides to try out the computer that was recently installed in her village. She comes to the computer and the avatar asks her question in English. Not knowing what to do or what the question is, she goes home to ask her brother for help. He has gone to school and knows some basic English, and has used the computer before. Together they return to the computer and he translates the question for her. 

He also shows her how to open the browser and how to copy and paste text. She copies the question into the Google Search bar and clicks "Images". She sees that this results in pictures and photographs. Her brother leaves to play cricket and she continues to use the computer. Every question the avatar asks she copy-pastes into Google and looks at the pictures. After multiple questions she starts to recognize certain English words.

This way she can (independently) learn some basic English with very little help. 

\section{Pilot project}

While working on the project implementation, we have decided to restrict ourselves with the pilot version of the student interface and to test in on a number of groups of Dutch children. The student interface involving a virtual assistant is the most challenging task from the UI point of view. The rest of the application is rather straightforward. With the pilot project we tried to understand whether the ideas of Sugata Mitra implemented with the help of the virtual assistant are indeed beneficial for the facilitation of group-based computer-asssisted learning.



\section{Testing}

\section{Challenges}

%how do different avatars affect the study process?

\section{Conclusions}

\begin{thebibliography}{99}
\bibitem{Cowell} Cowell.
\end{thebibliography}

\end{document}